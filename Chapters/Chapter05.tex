% Chapter 5

\chapter{Conclusions} % Chapter title

\label{ch:conclusion} % For referencing the chapter elsewhere, use \autoref{ch:name} 

%----------------------------------------------------------------------------------------
\section{Summary of work}
For the work outlined in the previous chapters in the thesis, a total count of 50 classes of 12 Java packages were implemented for the system which sums up to an approximate of 8,000 lines of code exclusive of blank lines. On top of that, a total number of 23 MATLAB scripts have been produced for the analytical work which include both the first stage point-based analysis using D-LDA method and the second stage area-wise multivariate analysis of the integrated knowledge base. In addition, a user manual for the java package \emph{au.csiro.iekbase.integration} has been written and formally typesetted for giving instructions on how to set up a knowledge base for any arbitrary location within Australia. The implementation and analytical results for the first stage work of the thesis has been published on peer-reviewed journal and presented in an IEEE Conference in Baltimore, MD, USA in November 2013\citep{Li2013}. The work carried out for the second stage analysis has novelties in regards to the approach it undertook for the water balance evaluation on a continental scale, of which the manuscript of a journal article has been drafted and due for peer-review and publication in the near future.
\section{Further development}
To extend from the work completed in the thesis, firstly, a further set of supervised machine learning algorithms can be trialed on the data set, potentially includes but not limited to \acp{hmm} which has proven effective especially for time series data and a few other neutral networks. Secondly, the current weekly directory structure of the integrated knowledge base has a flaw in its design which is that in the case where the a small portion of the data for analysis is required, for example, one variable of the AWAP data, such as the rainfall of a given location is needed, the entire data file needs to be loaded completely for the content to be extracted.  Also, modifications to the location of a defined case study is inefficient in the sense that a complete rerun needs to be performed for such alteration to take place.\\
\newline
A \ac{gui} can be developed in a similar manner that was employed for the point-based proof-of-concept demo, for the area-wise analysis to provide an easy-to-use digital platform for the farming practitioners to utilise thus fulfilling the initial objective as a irrigation water usage decision support system.\\
\newline
The proposed approach for heterogeneous knowledge integration in the thesis can be applied to fields of research other than water balance estimation which was the main focus of the outlined work. Given sufficient data of various independent data sources, it is theoretically viable to attempt to use one widely available input in terms of spatial and temporal resolutions as a proxy to an environmental attribute that is conventionally measured via field experimentation or empirical modelling, thus overcoming the drawbacks of having limited experimental results over a long timespan due to practical difficulties and/or financial constraints etc. Examples of such studies could include but not limited to: the human impact on the Great Barrier Reef ecosystem - a continuous study of human behaviour to the environment; the deforestation of the Amazon tropical forest in the South America - Is there a specific human-factor involved or it is merely a process of the Earth ecosystem's natural cycle.