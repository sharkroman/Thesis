% Chapter 1

\chapter{Introduction} % Chapter title

\label{ch:introduction} % For referencing the chapter elsewhere, use \autoref{ch:introduction} 
Australia, despite being the country having the 6th largest land area in the world, has water resource scarcity issues due to the fact that a large portion of the water bodies are located in regional areas. Whereas in the regions which has highly concentrated population and agricultural activities the water resources are heavily limited thus often regulated and  
has a premium pricing scheme. According to the latest Water Account Report released by \citet{Statistics2013}, it was found that a total amount of 75,000 \ac{gl} is approximated to be extracted and used within the Australian economy. Among which, the actual total consumption was 16,000 GL, a 20\% increase from the previous annum. The agricultural sector alone, accounted for 9,418 GL, or a majority of 59\% total water consumption, which was mainly used for irrigation.\\
\newline 
Due to the scarcity of water supply and highly priced electricity for operating irrigation systems as well as the fact that excessive water in the soil could cause fungal infection in the plants\citep{Wong1984}, how to properly devise an appropriate irrigation scheme has been a chronological issue that's yet to be resolved from the agricultural practitioners' perspective. Thus, a low-cost solution is needed to provide reliable advices on irrigation planning for the farmers in Australia so that the wastage and costs can be optimized.\\
\newline
A conventional approach is to combine field experiments that are conducted under controlled environment with root zone soil water balance modelling. However, field experiments are costly to conduct in terms of labour and monetary expenses, plus that it has its limitations on specific soil conditions which means that it would be impractical to generalise a water balance estimation model which can be applied elsewhere entirely based on field experiment results. There is a demand for on-request complementary knowledge integration and automatic interpretation of the knowledge that are based from multiple data sources to increase efficiency. \\
\newline
The thesis aims to investigate the possibility of producing most probable high resolution estimation regarding the water balance in any region within Australia by implementation of an intelligent system that can integrate spatio-temporal data from various independent sensors and models, based on a scientific data-driven approach rather than human intuition which nevertheless is the conventional approach employed empirically by farming practitioners.\\
\newline
The underlying scientific basis for the analysis outlined in the thesis is an application of Machine Learning based techniques within the domain of Artificial Intelligence. Machine Learning can be broadly defined as the study of systems that can learn from data. Due to the advancement of integrated circuit technology, which followed Moore's law, the computing power had boosted greatly in the last few decades, machine learning has proven to be an effective approach to analyze data on a large scale, and has been made useful for numerous applications, such as automated traffic classification\citep{Zander2005}, computerised facial expression recognition etc\citep{Bartlett2005}.\\
\newline
In the thesis, the study of water balance model as a measure of irrigation water usage is explained in \autoref{ch:WaterBalance}. A series of water balance models have been discussed with formulas and case studies as support materials for each of the model, which is followed by the derivation and discussion of the water balance model that is used for the analytical work in the thesis.\\
\newline
\autoref{ch:dataintegration} discusses about each of the data set used in the thesis, who are the data providers, what are their respective scopes and objectives, what are the availability of the data in terms of spatial and temporal specifications, what data formats are available and via what channel etc. This chapter also explains the implementation aspects of work involved for the thesis, including accessing the data via different platform and method, re-projection of specific geo-referencing coordinate system to a uniform system and pre-processing the data to a uniform weekly structure for the ease of analysis at later stages.\\ 
\newline
In \autoref{ch:study}, a point-based approach is first implemented and discussed. Of which, a novel knowledge and machine learning analysis based water usage recommendation system has been investigated and proposed based on the \ac{csiro} sensor cloud. To demonstrate the proposed approach to key stakeholders of the project, an Android \ac{os} based recommendation framework is developed as a proof-of-concept product. Based on this preliminary proof-of-concept work, an area-wise approach is then implemented and discussed, which includes three different algorithms that follow an progressive order, i.e. the latter ones were developed on the basis of the previous ones such that the results can be improved.\\
\newline
The conclusion chapter, \autoref{ch:conclusion} summarises the analytical results and the implementation work that are done for the thesis. It also talks about the potential applications and improvements that can be done on the basis of the work that are discussed.\\
\newline
\autoref{ch:appendix} includes a manual that was produced by me, the author of the thesis, for the purpose of giving instructions on using the knowledge integration system that were developed by me for the work in this thesis. Also, a complete list of the Java classes and MATLAB scripts implemented for the work that are outlined in the thesis is provided in \ref{listofjava} and \ref{listofmatlab} respectively.