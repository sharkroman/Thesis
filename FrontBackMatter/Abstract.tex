% Abstract

\pdfbookmark[1]{Abstract}{Abstract} % Bookmark name visible in a PDF viewer

\begingroup
\let\clearpage\relax
\let\cleardoublepage\relax
\let\cleardoublepage\relax

\chapter*{Abstract} % Abstract name

Australia has an uneven distribution of water resources across the continent. There is a relative abundance in the north where few people resides whereas in the densely-populated southern regions water scarcity has been a long-lasting issue for the past decades, especially in the agricultural sector where irrigation is the lifeline of the industry. This thesis proposes and investigates a novel approach of using data-driven machine learning technique to overcome the drawbacks of conventional water balance modelling approaches where the spatial and temporal resolution are limited and rely highly on condition-specified field experiments as a control group. The proposed approach integrates various independent data sources of meteorological information as a knowledge base, then conducts a two-stage analysis of the data using numerous algorithms to trial and error for the optimal result. The first stage consists of the implementation of a computing cloud based central server and an Android client-end application which delivers the analytical results to the users, as a proof of concept demo. The second stage added Landsat and \ac{modis} which are provided by \ac{nasa}. The \ac{vi} products of MODIS mission is used as a proxy for irrigation water balance in machine learning analysis which is proven to be credible and valid for practical applications, for instance, an enhanced resolution water balance index surface is produced from trained supervised machine learning models using the MODIS VI data as the input. The findings in the thesis indicate that given sufficient data, the same data-driven approach can be adapted for studies of other meteorological fields different from irrigation water usage.

\acresetall

\endgroup			

\vfill